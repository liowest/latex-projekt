\documentclass[a4paper,12pt]{article}
\usepackage[utf8]{inputenc}
\usepackage{graphicx}
\usepackage{lipsum}
\usepackage{hyperref}
\usepackage{geometry}
\usepackage{pgfgantt}
\geometry{a4paper, margin=2.5cm}

\title{Zusammenfassung der Übungen des Wintersemesters}
\author{  Lionel Westreicher, Moritz, Aron}
\date{\today}

\begin{document}

\maketitle

\tableofcontents

\newpage

\section{Projektidee: Wetter-gestützte To-Do-Liste}

\section{Zielsetzung}
Das Projektziel ist die Entwicklung einer wettergestützten To-Do-Liste, die es den Benutzern ermöglicht, Aufgaben basierend auf aktuellen Wetterdaten effizient zu planen und zu verwalten. Die App soll Funktionen zur Verwaltung von Aufgaben (Hinzufügen, Löschen) bieten, Wetterinformationen in Echtzeit für den aktuellen oder einen benutzerdefinierten Standort abrufen und relevante Wetterwarnungen anzeigen. Zusätzlich bietet die App eine tägliche Übersicht über anstehende Aufgaben und das aktuelle Wetter, um die Tagesplanung zu optimieren.

\section{Projektorganisation}
Für das Projekt wird eine einfache Projektorganisation gewählt. Der Projektauftraggeber (Prof. Netzter) übernimmt die Rolle der übergeordneten Instanz, die die Gesamtziele definiert und überwacht. Der Projektleiter (Lionel) koordiniert die tägliche Arbeit, verteilt Aufgaben und stellt sicher, dass die Projektmeilensteine erreicht werden. Die Projektmitarbeiter (Moritz und Aron) und der Projektleiter (Lionel) setzen die technischen und entwicklungsbezogenen Aufgaben um. Der Kunde (Klasse) ist der Endbenutzer der App, der Feedback zur Benutzerfreundlichkeit und Funktionalität gibt.

\section{Funktionen des Projekts}
\begin{itemize}
    \item \textbf{Aufgabenverwaltung}: Aufgaben hinzufügen, bearbeiten und löschen.
    \item \textbf{Wetterintegration}: Wetterdaten abrufen und für Aufgaben berücksichtigen.
    \item \textbf{Wetterwarnungen}: Relevante Hinweise basierend auf Wetterbedingungen.
    \item \textbf{Tägliche Übersicht}: Anzeige der Aufgaben des Tages und aktuelles Wetter.
\end{itemize}

\section{Technologien}
\begin{itemize}
    \item \textbf{Dateispeicherung}: JSON-Dateien oder Textdateien.
    \item \textbf{API}: OpenWeatherMap oder eine ähnliche Wetter-API.
\end{itemize}

\section{Projektstrukturplan}
Der Projektstrukturplan wurde auf GitHub bereitgestellt.

\section{Gantt Chart}
Der detaillierte Ablaufplan des Projekts, einschließlich Meilensteinen und Abhängigkeiten, ist auf GitHub verfügbar. Der Gantt-Plan hilft dabei, Verzögerungen zu vermeiden und den Projektverlauf visuell darzustellen. Bitte besuche das [GitHub-Repository](https://github.com/dein-repository) für weitere Details.

\section{Projektumfeldanalyse}
\begin{figure}[h]
    \centering
    \includegraphics[width=0.8\textwidth]{umfeldanalyse1.png}
    \includegraphics[width=0.8\textwidth]{Umfeldnalyse.png}
    \caption{Projektumfeldanalyse}
    \label{fig:projektumfeldanalyse}
\end{figure}

\section{Risikoanalyse}
\begin{figure}[h]
    \centering
    \includegraphics[width=0.8\textwidth]{risikoanalyse1.png}
    \includegraphics[width=0.8\textwidth]{risikoanalyse.png}
    \caption{Risikoanalyse}
    \label{fig:risikoanalyse}
\end{figure}

\section{Git-Dokumentation}
Das Projekt wird in einem Git-Repository versioniert. Jede Änderung wird durch aussagekräftige Commits dokumentiert, um eine nachvollziehbare Historie sicherzustellen.

\end{document}
